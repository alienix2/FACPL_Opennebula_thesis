% !TEX root = ../Thesis.tex

\chapter{Conclusioni e sviluppi futuri}
In questo capitolo verranno analizzati i punti positivi e le criticità riscontrabili nella modalità di utilizzo di FACPL per un progetto come quello in esame. Ci si concentrerà quindi sui pregi e difetti che caratterizzano la libreria Java con cui viene distribuito. Saranno inoltre proposti alcuni sviluppi futuri per migliorare la libreria e anche per sviluppare ulteriormente i progetti ideati in questa tesi.

\section{Criticità e punti di forza della libreria Java di FACPL}
Per questo progetto le prime criticità sono state quelle riguardanti la poca diffusione di FACPL e la conseguente mancanza di documentazione molto dettagliata. Capire il funzionamento della libreria non è stato immediato dato che anche la documentazione ufficiale non è completa. Non è spiegato all'effettivo come implementare le componenti del sistema che devono per forza essere modificate a mano come mostrato nella sezione \ref{sec:modalitaFACPL}.\par
È anche vero però che una volta compreso come interagiscono fra loro le classi della libreria, inserire le funzionalità già sviluppate è stato molto semplice. Questo ci porta a considerare che in effetti la libreria è ben sviluppata ma manca soltanto di una documentazione un po' più specifica che indichi come integrare il software in un progetto più ampio; in particolar modo servirebbe una guida su come gestire le variabili di sistema e le \emph{PEPAction}.\par
Un'altra mancanza è una modalità semplice di conversione delle policy e delle richieste scritte in FACPL in codice Java. Sono documentati dei validatori e convertitori del linguaggio FACPL che però, come indicato più volte anche in questa tesi, sono utilizzabili tramite la UI di Eclipse. Nella libreria sono presenti delle classi di validazione e conversione di file da FACPL verso diversi linguaggi fra cui anche Java, tuttavia non è presente alcun tipo di documentazione al riguardo; guardando la letteratura si fa sempre riferimento ai convertitori automatici accessibili da Eclipse. Una strada percorribile molto facilmente per migliorare questo aspetto, sarebbe quella di consigliare l'utilizzo dello \emph{StandaloneGenerator} fornito come esempio.\par
Un altro aspetto migliorabile è la gestione delle dipendenze, usare un gestore come Maven renderebbe la libreria più appetibile per un utilizzatore e permetterebbe di rendere la libreria più mantenibile e più semplice da aggiornare.\par
L'ultima problematica riguarda la dipendenza da Java 8, in realtà questo problema è facilmente risolvibile. Ad esempio durante il nostro  sviluppo per un periodo si è compilato il codice con Java 11 senza problemi con qualche piccolo accorgimento. Tuttavia per evitare di dover svolgere test estensivi su parti di codice già testate con Java 8 si è dovuto ritornare ad utilizzare questa versione.\par
Parlando degli aspetti positivi troviamo sicuramente la facilità di integrazione con un progetto già esistente. Quando si utilizza FACPL un approccio che funziona bene è quello di pensare alla logica della propria implementazione in modo del tutto distaccato da FACPL e poi integrare le due parti. Per farlo basta scrivere del codice che fornisca due cose:
\begin{itemize}
    \item Dei metodi in grado di ritornare delle informazioni sullo stato del sistema
    \item Delle classi che permettano di eseguire delle azioni sul sistema e che devono aderire all'interfaccia \texttt{IPepAction}
\end{itemize}
A quel punto cambiando le classi \texttt{ContextStub\_Default} e \texttt{PepAction}, in poche righe il proprio codice è utilizzabile dal sistema FACPL.\par
L'altro aspetto molto positivo è la gestione del logging, infatti questo è già molto espressivo e permette di capire in modo dettagliato le scelte che vengono fatte dal sistema. Questo torna utile in fase di test del proprio codice, oltre che per un eventuale utilizzatore finale.\par

\section{Sviluppi futuri}
Per quanto riguarda la libreria Java di FACPL uno sviluppo futuro potrebbe riguardare l'aggiornamento ad una versione di Java LTS più recente come la 11 o la 17. Questo permetterebbe innanzitutto di rimuovere alcune dipendenze da librerie esterne, che sono entrare a far parte delle librerie standard nelle versioni più recenti di Java. Inoltre renderebbe la libreria più adatta ad essere integrata con codice più recente.\par
L'altro aspetto già largamente discusso sarebbe una nuova modalità di gestione delle dipendenze, che renderebbe sicuramente più facile anche eseguire il passo descritto sopra.\par
Per concludere, una volta aggiornata la libreria, sarebbe utile migliorare la documentazione e mostrare un workflow consigliato da seguire in dei progetti più grandi come quello considerato in questa tesi.\par
Per quanto riguarda invece i due progetti ideati in questa tesi, sicuramente ci sono diverse possibilità di ampliazione. Si potrebbe pensare di integrare molte più \emph{PEPAction} riguardanti OpenNebula. Questo passo non sarebbe difficile da fare dato che è già presente la classe astratta che può essere estesa e da cui partire per scrivere nuovi comandi. Alcuni comandi interessanti che potrebbero essere aggiunti sono il cambiamento di rete per una virtual machine o il cambiamento delle impostazioni di una virtual machine a runtime, entrambe azioni facilmente implementabili con le API di OpenNebula.\par

Una miglioria che si potrebbe portare avanti riguarda il front-end e l'integrazione di un sistema di autenticazione. Questa parte nella logica attuale è rimandata alla persona che intende inserire il codice nella sua infrastruttura. Il sistema di autenticazione potrebbe tornare utile per un utilizzatore finale che vuole partire da zero. Si considera però che nella maggior parte dei sistemi la nostra web-app sarà gestita tramite un sistema di autenticazione generale, che racchiude anche altre web-app. Il front-end è sicuramente migliorabile anche in molti altri modi, espressi nel capitolo \ref{cap:capitolo3}.\par